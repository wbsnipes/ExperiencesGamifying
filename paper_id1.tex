% This is "sig-alternate.tex" V2.0 May 2012
% This file should be compiled with V2.5 of "sig-alternate.cls" May 2012
%
% This example file demonstrates the use of the 'sig-alternate.cls'
% V2.5 LaTeX2e document class file. It is for those submitting
% articles to ACM Conference Proceedings WHO DO NOT WISH TO
% STRICTLY ADHERE TO THE SIGS (PUBS-BOARD-ENDORSED) STYLE.
% The 'sig-alternate.cls' file will produce a similar-looking,
% albeit, 'tighter' paper resulting in, invariably, fewer pages.
%
% ----------------------------------------------------------------------------------------------------------------
% This .tex file (and associated .cls V2.5) produces:
%       1) The Permission Statement
%       2) The Conference (location) Info information
%       3) The Copyright Line with ACM data
%       4) NO page numbers
%
% as against the acm_proc_article-sp.cls file which
% DOES NOT produce 1) thru' 3) above.
%
% Using 'sig-alternate.cls' you have control, however, from within
% the source .tex file, over both the CopyrightYear
% (defaulted to 200X) and the ACM Copyright Data
% (defaulted to X-XXXXX-XX-X/XX/XX).
% e.g.
% \CopyrightYear{2007} will cause 2007 to appear in the copyright line.
% \crdata{0-12345-67-8/90/12} will cause 0-12345-67-8/90/12 to appear in the copyright line.
%
% ---------------------------------------------------------------------------------------------------------------
% This .tex source is an example which *does* use
% the .bib file (from which the .bbl file % is produced).
% REMEMBER HOWEVER: After having produced the .bbl file,
% and prior to final submission, you *NEED* to 'insert'
% your .bbl file into your source .tex file so as to provide
% ONE 'self-contained' source file.
%
% ================= IF YOU HAVE QUESTIONS =======================
% Questions regarding the SIGS styles, SIGS policies and
% procedures, Conferences etc. should be sent to
% Adrienne Griscti (griscti@acm.org)
%
% Technical questions _only_ to
% Gerald Murray (murray@hq.acm.org)
% ===============================================================
%
% For tracking purposes - this is V2.0 - May 2012

\documentclass{sig-alternate}
% *** CITATION PACKAGES ***
%
\usepackage{cite}
% cite.sty was written by Donald Arseneau
% V1.6 and later of IEEEtran pre-defines the format of the cite.sty package
% \cite{} output to follow that of IEEE. Loading the cite package will
% result in citation numbers being automatically sorted and properly
% "compressed/ranged". e.g., [1], [9], [2], [7], [5], [6] without using
% cite.sty will become [1], [2], [5]--[7], [9] using cite.sty. cite.sty's
% \cite will automatically add leading space, if needed. Use cite.sty's
% noadjust option (cite.sty V3.8 and later) if you want to turn this off.
% cite.sty is already installed on most LaTeX systems. Be sure and use
% version 4.0 (2003-05-27) and later if using hyperref.sty. cite.sty does
% not currently provide for hyperlinked citations.
% The latest version can be obtained at:
% http://www.ctan.org/tex-archive/macros/latex/contrib/cite/
% The documentation is contained in the cite.sty file itself.






% *** GRAPHICS RELATED PACKAGES ***
%
%\if CLASSINFOpdf
  \usepackage{graphicx}
  \usepackage{epstopdf}
 % \usepackage{auto-pst-pdf}
  % declare the path(s) where your graphic files are
  % \graphicspath{{./eps}{./pdf/}{./jpeg/}}
  % and their extensions so you won't have to specify these with
  % every instance of \includegraphics
  \DeclareGraphicsExtensions{.eps,.pdf,.jpeg,.png,.ps}
%\else
  % or other class option (dvipsone, dvipdf, if not using dvips). graphicx
  % will default to the driver specified in the system graphics.cfg if no
  % driver is specified.
 % \usepackage[dvips]{graphicx}
  % declare the path(s) where your graphic files are
  % \graphicspath{{./eps/}}
  % and their extensions so you won't have to specify these with
  % every instance of \includegraphics
 % \DeclareGraphicsExtensions{.eps}
%\fi
% graphicx was written by David Carlisle and Sebastian Rahtz. It is
% required if you want graphics, photos, etc. graphicx.sty is already
% installed on most LaTeX systems. The latest version and documentation can
% be obtained at: 
% http://www.ctan.org/tex-archive/macros/latex/required/graphics/
% Another good source of documentation is "Using Imported Graphics in
% LaTeX2e" by Keith Reckdahl which can be found as epslatex.ps or
% epslatex.pdf at: http://www.ctan.org/tex-archive/info/
%
% latex, and pdflatex in dvi mode, support graphics in encapsulated
% postscript (.eps) format. pdflatex in pdf mode supports graphics
% in .pdf, .jpeg, .png and .mps (metapost) formats. Users should ensure
% that all non-photo figures use a vector format (.eps, .pdf, .mps) and
% not a bitmapped formats (.jpeg, .png). IEEE frowns on bitmapped formats
% which can result in "jaggedy"/blurry rendering of lines and letters as
% well as large increases in file sizes.
%
% You can find documentation about the pdfTeX application at:
% http://www.tug.org/applications/pdftex

% *** ALIGNMENT PACKAGES ***
%
%\usepackage{array}
% Frank Mittelbach's and David Carlisle's array.sty patches and improves
% the standard LaTeX2e array and tabular environments to provide better
% appearance and additional user controls. As the default LaTeX2e table
% generation code is lacking to the point of almost being broken with
% respect to the quality of the end results, all users are strongly
% advised to use an enhanced (at the very least that provided by array.sty)
% set of table tools. array.sty is already installed on most systems. The
% latest version and documentation can be obtained at:
% http://www.ctan.org/tex-archive/macros/latex/required/tools/
\usepackage{balance}
\usepackage{mdframed}
%\usepackage{mdwmath}
%\usepackage{mdwtab}
% Also highly recommended is Mark Wooding's extremely powerful MDW tools,
% especially mdwmath.sty and mdwtab.sty which are used to format equations
% and tables, respectively. The MDWtools set is already installed on most
% LaTeX systems. The lastest version and documentation is available at:
% http://www.ctan.org/tex-archive/macros/latex/contrib/mdwtools

\begin{document}
%
% --- Author Metadata here ---
\conferenceinfo{WOODSTOCK}{'97 El Paso, Texas USA}
%\CopyrightYear{2007} % Allows default copyright year (20XX) to be over-ridden - IF NEED BE.
%\crdata{0-12345-67-8/90/01}  % Allows default copyright data (0-89791-88-6/97/05) to be over-ridden - IF NEED BE.
% --- End of Author Metadata ---

\title{Effectiveness of Instant Feedback on Developer Navigation Practices}
\subtitle{Does Instant Feedback Make Us Better Developers?}
%
% You need the command \numberofauthors to handle the 'placement
% and alignment' of the authors beneath the title.
%
% For aesthetic reasons, we recommend 'three authors at a time'
% i.e. three 'name/affiliation blocks' be placed beneath the title.
%
% NOTE: You are NOT restricted in how many 'rows' of
% "name/affiliations" may appear. We just ask that you restrict
% the number of 'columns' to three.
%
% Because of the available 'opening page real-estate'
% we ask you to refrain from putting more than six authors
% (two rows with three columns) beneath the article title.
% More than six makes the first-page appear very cluttered indeed.
%
% Use the \alignauthor commands to handle the names
% and affiliations for an 'aesthetic maximum' of six authors.
% Add names, affiliations, addresses for
% the seventh etc. author(s) as the argument for the
% \additionalauthors command.
% These 'additional authors' will be output/set for you
% without further effort on your part as the last section in
% the body of your article BEFORE References or any Appendices.

\numberofauthors{3} %  in this sample file, there are a *total*
% of EIGHT authors. SIX appear on the 'first-page' (for formatting
% reasons) and the remaining two appear in the \additionalauthors section.
%
% author names and affiliations
% use a multiple column layout for up to three different
% affiliations
\author{
\alignauthor
Will Snipes
\affaddr{ABB Corporate Research}\\
\affaddr{Industrial Software Systems}\\
\affaddr{Raleigh, NC USA}\\
\email{will.snipes@us.abb.com}
\and
\alignauthor
Anil R. Nair
\affaddr{ABB Corporate Research}\\
\affaddr{Industrial Software Systems}\\
\affaddr{Bangalore, India}\\
\email{anil.nair@in.abb.com}
\and
\alignauthor
Emerson Murphy-Hill
\affaddr{North Carolina State University}\\
\affaddr{Department of Computer Science}\\
\affaddr{Raleigh, N.C. USA}\\
\email{emerson@csc.ncsu.edu}\\
}



\maketitle
\begin{abstract}
As software development practices evolve, we face the continuous challenge of communicating new practices and tools to the community.  As developers, we both like to try new things and like to stick with the familiar.  In addition to training, discussing, and presenting software engineering practices and tools, we seek to communicate them through active two-way information embedded in the Integrated Development Environment (IDE).  While communicating, we also collect data on new and current practices and tools used by each developer by recording actions in the IDE.  With a rich data source, we evaluate what practices are used, define metrics that give individual feedback, and create an environment where developers can see how they are doing compared with their teammates.  

This approach helps understand practice and tool use over the long term and judges the lasting change a communication effort makes in everyday habits.  Tool support provides a mechanism to embed into Visual Studio an extension that provides the capture of usage data while giving feedback to users on their practices both in general and specific to the ones being promoted.  Through a points score system that assigns points to specific commands and tools we encourage everyone to use the new practices.  Providing a light score ranking view should help us stay motivated as we compare our command usage to our peers.  To demonstrate how this works, we implement a longitudinal study that demonstrates the results from this approach as we engage with developers over a month-long period.  

Results from this demonstrative study show that in general the approach is <blank> with a few developers <blank> of the study.  The effectiveness of this approach for communicating about practices is demonstrated through the continued use of the commands as we communicate them.  Evaluation of metrics for navigation ratio over time show that developer use the commands <blank> than they did during the benchmarking phase.  The result of the study shows that this method for communicating and promoting software engineering tools and practices <blank>.
\end{abstract}

% A category with the (minimum) three required fields
\category{H.4}{Information Systems Applications}{Miscellaneous}
%A category including the fourth, optional field follows...
\category{D.2.8}{Software Engineering}{Metrics}[complexity measures, performance measures]

\terms{gamification}

\keywords{empirical, training, learning, usability}

\section{Introduction}
%update
Deploying these techniques requires communicating and motivating developers to use them.  Murphy-Hill and colleagues  \cite{wbsnipes:Hill2011Peer} show that developers can effectively learn about new tools from their peers by watching them work and through recommendations, yet this happens very rarely, in part because developers rarely work on the same computer at the same time.  This personal connection limits the knowledge transfer to the co-located circle of colleagues connected with the early adopters.  
%update
Finding a way to automate the communication of technique recommendations between developers enhances deployment of best practices in a globally distributed software development organization.   
%PF the following sounds redundant
To best match the peer recommendation scenario described above, the method should mimic the personal recommendation model while rewarding successful adopters.

\section{Study Design}
Creating a longitudinal study involves defining a long-term goal that is measurable in increments, determining the discrete steps that the study will use to communicate the goal to the participants, distribute interventions to participants, and track participants progress over time.  Many software engineering studies are conducted at discrete points in time via surveys of sampled participants.  Perhaps prior to an intervention, and multiple points following an intervention.  This study demonstrates a method to perform an intervention, continuously monitor the progress of participant response to that intervention, and provide continuous feedback to the participant and researchers about the progress.  Continuous meaning that no information gaps exist in the data collection, there are no periods of unmonitored activity by the participant.  Continuous also applies to the intervention being evaluated, where the participant and researchers receive constant feedback about the conformance to the intervention's purpose.  Studying an intervention in a continuous context determines how sticky the intervention is in the mind of the developer.  It can remove bias that might be introduced by periodic surveys or audits and enhance the habits formed in the participant receiving the intervention.

Designing the study with continuous measurement but no control group presents a traditional study design with some challenges.  We overcame this by dividing the study in to phases where the participants themselves form a control group.  The first phase baselines their activities determining the current state of their practices.  The second phase presents the intervention to the participants.  This phase the participants do not receive continuous feedback on their compliance with the intervention.  This phase simulates the traditional approach of communicating an intervention then returning with an audit to determine whether participants are complaint with it.   The last phases turns on continuous feedback to the participants on their compliance with the intervention.  As they receive the feedback, the participants can determine for themselves how much they are implementing the intervention.  In the last phase they possess complete knowledge of the intervention and their compliance with it making it plausible that they can continuously control their compliance.

Developing the study with the available tooling means that we follow steps as follows.
1. Define the practice we wish to measure compliance with
2. Develop a mapping of the practice to actions that are captured by the blaze tool from the IDE
3. Weight the actions according to their importance to the practice and assign points to the actions in the configuration file
4. Develop the intervention method
5. Determine the phases and length of the phase in the study according to baseline, intervention, and continuous feedback
6. Select participants to deploy
7. Define a pre-study questionnaire if one is necessary to collect associated information
8. Define a post-study questionnaire if necessary to collect post study information
9. Communicate with the participants about the study
10. Execute the study and report results

\subsection{Setup the practice}
Selecting a practice to study involves defining specific characteristics of the practice and how capturing actions in Visual Studio relates to the practice.  In this study we selected the structured navigation practice shown by Robillard et. al. as correlated with effective program maintenance \cite{wbsnipes:Robillard2004How} .  Studying structured navigation is a natural fit for capturing commands issued in the IDE.  The first step is to create the classification of commands that are involved in navigation.  Then differentiate which commands we consider as structured navigation from unstructured navigation.  

The Blaze tool provides an XML configuration file that allows defining multiple levels of categories for commands thus we can categorize commands that are navigation then further classify them into structured and unstructured navigation.
We classified commands as structured navigation when they allow the developer to locate a code element following the program structure from another element.  For example, Navigate To \textit{Ctrl+,}, Go To Definition \textit{F12}, and Find All References \textit{Ctrl+K,R} are commands that follow structure of the code that we classify as structured navigation.
Commands that we considered as unstructured navigation include scrolling through a file using keys or the scroll-bar, opening files from the Solution Explorer, Find commands (e.g. Find in Files), Go-to line number, and children of these commands (e.g. Find Next).

After classifying the commands we assigned points to the commands in the same configuration file.  Structured navigation commands received 1-10 points each while unstructured navigation commands received 0 points.  Assigning negative points while possible, is discouraged as this may demotivate folks who do not have detailed knowledge of how the study is designed.  When assigning points, make sure you recognize the relationship between points and levels such that "leveling up" happens at a frequent enough pace to make that motivating for users during the study.  Ideally users would level up after a working day or two worth of activity initially extending to a week or even a month following an exponential curve \cite{}.  Designing the points and levels such that above average developers can complete all levels during the study period is ideal.

\subsection{Study Intervention and Participants}

After selecting the practice to focus on, we needed to select an intervention method.  The term intervention, indicates that we are seeking to influence developers to perform a practice previously defined by software engineering community.  Intervention involves communication about the practice in a way that motivates adoption \cite{}.  Interventions can include presentations, one-on-one discussions, distance education mechanisms such as webinars, and word of mouth recommendations.  Part of the research objective is to determine whether we can acheive adoption of a practice with alternative interventions that are more deliberate than word of mouth and perhaps more scalable than webinars or presentations.  Utilizing the automation capabilities of Blaze, we chose to automatically pop-up a window with training information on structured navigation practices and tools. For this study we chose to make a nice looking graphic that users click on to see information about navigation commands, obtain and install the Sando search tool, and learn about the Blaze game.  The site uses a combination of video and blog-post style information easily accessible to the developer.  The usefulness of the communication site is rated as part of our study completion survey.

The study established three time periods, a prelinary period to establish a baseline, a period where information is communicated but no immediate feedback or points are displayed, and a period where points are displayed and all information is available.  We chose to deploy the Blaze tool without communicating the study purpose for the baseline period so that an accurate picture of the baseline could be captured by the tool.  Then the tool would pop-up a display with the information on the communication web site asking the developers to consume the information and begin using the practices.   Finally the display of points fo accumulated by the developer provides the instant feedback mechanism.  

We conducted this study with an intact team of developers working in an R\&D facility in India on a large industrial software system.  Developers were asked to volunteer for the pilot by their management, however, managers did not know who participated in the study.  We intended to keep managers out of the loop to reduce concerns among the developers that the data may be used for purposes other then the intended study.  By selecting an intact team, we hope to leverage aspects of motivation such as the Leaderboard to spur feelings of competition among teammates.  

\section{Related Work}
%update and expand
Researchers have developed systems that aim to answer specific questions using data from command logs. Zorro \cite{V:Johnson2007Automated} leverages the HackyStat data to determine if developers are adhering to test-driven development.  The tool employs a similar approach dividing the work of the developer into episodes delineated by when test cases pass.  The researchers define multiple types of episodes then determine whether the developer uses Test-Driven-Development or not in each episode.  

Murphy-Hill, Parnin, and Black \cite{V:MurphyHill2012How} use the Mylyn Monitor to explore whether or not developers use the automated refactoring tools present in Eclipse.  They look for specific refactoring commands in Eclipse and determine the amount of time developers use tools versus hand refactoring the code.  For this study we build a tool that captures the use of all commands within the Visual Studio environment including refactoring, navigation, and edit commands.

Robillard, Coelho and Murphy explore hypotheses around how developers can be more effective at performing a maintenance task \cite{wbsnipes:Robillard2004How}.  Key conclusions are that developers are more more successful finding and fixing bugs when they create a detailed plan for implementing a change, use structured navigation with keyword search or cross-reference search, and only review methods once during their search.  This builds on this work by testing methods for increasing the use of structured navigation in developers everyday practice.

Singer and Schneider demonstrate the use of a message board and points for encouraging students to increase their frequency of commits to the source code repository.\cite{Singer2012It}  The communication mechanism enabled students to see each others' progress and resulted in more frequent commits than baseline.  Participants valued the communication and collaboration aspect and some valued the competition enough to change their opinion on optimum commit frequency.  The subsequent thesis by Singer \cite{Singer2013a} describes results of an experiment conducted across two iterations of a class where active feedback on commits was deployed to one course and another course served as the control group where commit frequency was simply monitored.  Results show an increase in the frequency of commits at a statistically significant level.  This study uses finer grained instrumentation to identify practices being used between commits such as structured navigation.

The Pro Metrics (PROM) tool provides a framework for collecting data for further analysis from tools used by developers.\cite{Coman2009Casestudy}  It provides a flexible data model and a plug-in architecture to facilitate collection from different data sources.  Studies conducted using prom include a series of studies by it's creators on trends in time spent Pair Programming \cite{Coman2008Investigating}, benefits of refactoring on productivity \cite{Moser2008Case}, impact of refactoring on re-usability \cite{Moser2006Does}, and prediction of effort \cite{Abrahamsson2007Effort}.  The PROM tool and studies that use it track the time spent editing files and methods from the IDE and code metrics obtained trough source code analysis.  Applying these data to specific research questions such as whether refectoring improves productivity \cite{Moser2008Case} shows the utility of combining automated effort data with code metric data.  Taking a different approach to focus on techniques, our study used more detailed events captured from the IDE  to detect how the user was navigating through the code in addition to knowing how long the editor is open for a particular file.

Building on this success, this work proposes a generalized method to automate the identification of the techniques described in the related work and others, and determine when the techniques would help a developer improve their work patterns.  Then we intend to create a game-like virtual environment where developers learn from each other and are motivated to adopt the best practices developed by these and other researchers.

\section{Conclusions}
This paragraph will end the body of this sample document.
Remember that you might still have Acknowledgments or
Appendices; brief samples of these
follow.  There is still the Bibliography to deal with; and
we will make a disclaimer about that here: with the exception
of the reference to the \LaTeX\ book, the citations in
this paper are to articles which have nothing to
do with the present subject and are used as
examples only.
%\end{document}  % This is where a 'short' article might terminate

%ACKNOWLEDGMENTS are optional
\section{Acknowledgments}
This section is optional; it is a location for you
to acknowledge grants, funding, editing assistance and
what have you.  In the present case, for example, the
authors would like to thank Gerald Murray of ACM for
his help in codifying this \textit{Author's Guide}
and the \textbf{.cls} and \textbf{.tex} files that it describes.

%
% The following two commands are all you need in the
% initial runs of your .tex file to
% produce the bibliography for the citations in your paper.
\bibliographystyle{abbrv}
\balance
\bibliography{paper_id1} 
 % sigproc.bib is the name of the Bibliography in this case
% You must have a proper ".bib" file
%  and remember to run:
% latex bibtex latex latex
% to resolve all references
%
% ACM needs 'a single self-contained file'!
%

\end{document}
